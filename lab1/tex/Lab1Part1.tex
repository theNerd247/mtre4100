\documentclass[main.tex]{subfiles}
\begin{document}
	\section[Lab Development]{Lab Development}
	\subsection[First Order Analysis]{First Order Analysis}
	
	First order analysis begins by analyzing a simple lowpass Butterworth filter shown in the next figure.
	%butterworth diagram here?
	\\
	The voltage drop across the resistor is given by the equation
	\[v_{r} = iR\] and the current across the capactitor is 
	\[i_{c} = C \frac{d v_{c}}{dt}\]
	\\Kirchoff's voltage law is expressed as
	\[v_{0} = v_{r} + v_{c}\]
	\\
	Substituting the first two equation into the third results in the following differential equation
	\[v_{0} = RC \frac{d v_{1}}{dt} + v_{1}\]
	\\where the constant \(RC\) is the time constant and has units of seconds as demonstrated.
	
	
	\[RC = (\frac{M * L^2}{T^2 * I^2} )(\frac{T^4 * I^2}{M * L^2}) = T\]
	\\
	For the case of \(R = 100\Omega\) and \(C = 0.1061 mF\) the time constant, \(RC = 10.61 ms\). For the case of \(R=50\Omega\) and \(C=0.1061mF\) the time constant \(RC=5.305ms\).
	
	\section{Experimental Analysis}
	The lab "experiment" was conducted using LabView. First, a signal generator block was placed and configured to produce a 1Hz square wave. The generator output was then connected to a filter block configured to simulate a first order lowpass Butterworth filter with a cutoff frequency of 30 Hz. Finally, both the output of the signal generator and the output of the filter were connected to a graph block so both the original and filtered waves could be seen.
	
	\section{Mathematical verses Experimental Analysis}
	
	
\end{document}