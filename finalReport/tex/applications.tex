\documentclass[main.tex]{subfile}

\begin{document}

\section{Applications of SBMPC} 
\label{sec:applications_of_sbmpc}

\cite{autoVehicle,auv} show applications of SBMPC in the path-planning problem
for mobile robots. In these applications the robot is driven by a SBMPC control
algorithm that enables optimized path planning to move the robot towards a goal
while avoiding detected objects. \cite{uphill} shows another modern application.
This project solves the issue of path planning for a robot going uphill. The
issue is that the motor commands sent to the robot depend on the initial
position of the robot; if the robot doesn't have enough initial momentum then it
will not be able to reach the top of the hill. In a more generalized context
this project shows how SBMPC can be used to solve the local minimum avoidance
problem found in robot path planning.

Aside from currently implemented SBMPC algoriths we propose some other
applications for industrial settings. First we extend the application shown in
\cite{uphill} to path planning for serial robot arms that perform pick and place
motions. One issue in an industrial setting is the difficult task of
re-programming a robot if the product being picked change (such as size or
weight). In some circumstances it is necessary to change the robot entirely to
account for these changes in the product. With an SBMPC algorithm their are no
changes necessary for modifying the path planning of the robot arm - which saves
both time and money. The SBMPC algorithm will account for the changes made in
the product and, because it accounts for the surrounding environment,
automatically adjust the robot's path planning.

Industrial robotic applications are numerous. One import application of SBMPC is
object avoidance of dynamic environments - such as with other moving machinery
interacting with the robots. The integration point for SBMPC algorithm would be
for online path planning of the robot. The goal being of course to move the
robot in such a way to avoid detected objects in the workspace. 

Finally another proposed application of SBMPC in industrial settings were
high-speed MIMO controllers for production processes are required. While a
solution to this process is still being pursued (\cite{autoVehicle} mentions
this algorithm being slow in computation time) computation speed optimization
algoithms could be implemented. \cite{uphill,auv} discuss briefly that MPC
algorithms have been widely used in industial settings - mainly in chemical
processing plants - and compared to SBMPC is extremely slow. So while SBMPC
still needs optimization it still is a viable solution for some high-speed
applications. 

% section applications_of_sbmpc (end)

\end{document}
