\documentclass[main.tex]{subfile}

\begin{document}

\section{Applications of SBMPC} 
\label{sec:applications_of_sbmpc}

From \cite{autoVehicle,uphill,auv} we have three robotic applications for SBMPC.
All three are applied to path planning for mobile robots. Within such
applications the control variables are typically kinematic based such as
position and velocity. Current applications include generating optimized paths
to a goal for mobile robots.  According to \cite{auv} SBMPC is also suitable for
path planning autonomous underwater vehicles in searching out mines. In
\cite{uphill} we find that SBMPC is good for detecting and avoiding local
minimum in the system state and perform path planning accordingly.

Industrial robotic applications are numerous. One import application of SBMPC is
object avoidance of dynamic environments - such as with other moving machinery
interacting with the robots. The integration point for SBMPC algorithm would be
for online path planning of the robot. The goal being of course to move the
robot in such a way to avoid detected objects in the workspace. For example a
kinematic model of a $3$ prismatic joint robot we have the following kinematic
model: 

\begin{align}
	\vec{x}_{k+1} &= A\vec{x}_{k} + B\vec{u}_{k}
\\ \vec{y}_{k+1} &= C\vec{x}_{k} + \vec{d}
\end{align}
where $A$,$B$,$C$ form the state space model for the system. $\vec{x}$ is the
state of our system, $\vec{u}$ is the joint space acceleration, and $\vec{y}$ is
the position and velocities of the end-effector.

Along with the state space model the constraints of our system will be: 

\begin{align}
	\vec{u}_{\text{l}} \leq \vec{u}_{k} \leq \vec{u}_{\text{u}}
	\\C(\vec{y}) \leq 0
\end{align}
where $C(\vec{y})$ are the collision constraints on the end-effector. The
end-effector. Our fitness functions would consider the least amount of
additional acceleration required to move the end-effector towards a target while
avoiding known obstacles.

% section applications_of_sbmpc (end)

\end{document}
