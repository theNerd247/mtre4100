\documentclass[main.tex]{subfile}

\begin{document}

% document text goes here
\section{Introduction} 
\label{sec:introduction}

Most industrial robotic applications today involve repetative motion tasks, with
little change in the operations of the robots themselves. Typically they are
programmed to optimally perform a single task. However today, with the growing
rates of production and ranging environment types we find that robots are needed
in more dynamic environments. For example in \cite{amazon} Amazon has begun
seeking robots that can perform item selection in warehouses while avoiding
collision with objects within their path.  Consider also the cost of programming
robots to perform very specific tasks. For industrial settings where the product
may change from day to day - such as glass manufacturing - it is difficult to
constantly reprogram and calibrate a robot to conform its kinematic path
planning algorithms to suite environment or product constraints. 

Model predictive controllers (or MPCs) boast two solutions to the changing
environment and constraints problem. First, MPC algorithms allow us to use
kinematic models of the robot into the problem directly - this is more a legacy
system support to older control methods. Second, and more importantly, we can
incorporate system (both robotic and environmental) constraints on our control
model. According to \cite{autoVehicle} MPC has been mostly restricted to slower
responding systems such as chemical manufacturing facilities. Recent approaches
to autonomous vehicle applications have integrated MPC algorithms as a means of
path planning \cite{auv,uphill} and used improved optimization stage algorithms
(we will explain this further in \secref{sbmpc}) for generating flexible and
optimal controller outputs. This optimization algorithm is known as sample-based
model controller (SBMPC).

% section introduction (end)

\end{document}
