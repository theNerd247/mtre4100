\documentclass[main.tex]{subfile}

\begin{document}

\section{\Labview VI} 
\label{sec:labview_vi}

To begin building the labview module we first design our PID controllers. We use
the PID controller block and the transfer function to generate the block
diagrams:

\begin{figure}[H]
	\begin{center}
		\includegraphics[width=\linewidth]{pidController1.png}
	\end{center}
	\caption{}
	\label{fig:}
\end{figure}

Notice the feedback from the transfer function goes directly into the PID
controller. This is the normal mode of operation for our controller. Later we
will add a sub-diagram to generate signal noise and perform our filtering. The
set point is a user-interface (or UI) input.

Next we create a block to generate the PID controller tuning parameters. We use
a series of UI inputs to make tuning the controller easier later on.


% section labview_vi (end)

\end{document}
