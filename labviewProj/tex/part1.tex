\documentclass[main.tex]{subfile}

\begin{document}

\section{\Labview VI} 
\label{sec:labview_vi}

To begin building the labview module we first design our PID controllers. We use
the PID controller block and the transfer function (who's configuration is shown
in \figref{pidParams}) to generate the block diagrams:

\begin{figure}[h]
	\begin{center}
		\includegraphics[width=\linewidth]{pidController.png}
	\end{center}
	\caption{Basic PID controller with plant model}
	\label{fig:pidController}
\end{figure}

\begin{figure}[h]
	\begin{center}
		\includegraphics[width=\linewidth]{pidParams.png}
	\end{center}
	\caption{Transfer Function model configuration}
	\label{fig:pidParams}
\end{figure}

\begin{figure}[h]
	\begin{center}
		\includegraphics[width=\linewidth]{pidConsts.png}
	\end{center}
	\caption{PID Controller with PID Constants UI For Tuning}
	\label{fig:pidConsts}
\end{figure}


Notice the feedback from the transfer function goes directly into the PID
controller. This is the normal mode of operation for our controller. Later we
will add a sub-diagram to generate signal noise and perform signal filtering. The
set point is a user-interface (or UI) input to allow the user to control the
input signal; making the simulation a bit more realistic.

Next we create a block to generate the PID controller tuning parameters. We use
a series of UI inputs to make tuning the controller easier later on.

To tune the PID controller we start with $P=1, I=0, D=0$ and simulate a step
response (see \figref{pidTune1}).  Next we adjust $P$ such that there is a bit
of overshoot in the step response. Finally we use $I$ and $D$ to level out the
overshoot and make the step response closer to being critically damped (while
still maintaining an under-damped response). The resulting response is shown in
\figref{pidTune2} with $P=10, I=10, D=0.001$.

\begin{figure}[h]
	\begin{center}
		\includegraphics[width=\linewidth]{pidTune1.png}
	\end{center}
	\caption{Initial System Step Response}
	\label{fig:pidTune1}
\end{figure}

\begin{figure}[h]
	\begin{center}
		\includegraphics[width=\linewidth]{pidTune2.png}
	\end{center}
	\caption{Final System Step Response}
	\label{fig:pidTune2}
\end{figure}

After tuning the PID controller to have an under-damped response we then add
noise to the output response to simulate vibrations in the sensors. To do this
we had a sub-diagram which add noise to the incoming signal and runs it through
a Butterworth-filter to obtain a clean sensor signal. We connect the output from
our plant model to the input of the noise-filter diagram and the output to the
process variable of the PID controller as shown in \figref{noise1}.

The noise-filter diagram is shown in \figref{noise2}. Gaussian white noise is
generated and then filtered by a low-pass filter ($10\dem{Hz}$). The filtered
noise is then added to the incoming signal to simulate real-world modulation.

\begin{figure}[h]
	\begin{center}
		\includegraphics[width=\linewidth]{noise1.png}
	\end{center}
	\caption{PID Controller with Noise-Filter Sub-Diagram}
	\label{fig:noise1}
\end{figure}

\begin{figure}[h]
	\begin{center}
		\includegraphics[width=\linewidth]{noise2.png}
	\end{center}
	\caption{Noise-Filter Diagram}
	\label{fig:noise2}
\end{figure}

% section labview_vi (end)

\end{document}
