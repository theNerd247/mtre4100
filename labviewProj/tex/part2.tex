\documentclass[main.tex]{subfile}

\begin{document}

\section{Final Project Design} 
\label{sec:final_project_design}

The final design of the measurement system contains two PID controllers, one for
tempurature as shown in the previous section, and a second for pressure
measurement. The final diagram is shown in \figref{finalDiagram}. The output
from the pressure and tempurature are used to calculate a prediction of the
fluid flow rate (per \eqref{flowRate}) in the pipe.

\begin{figure}[h]
	\begin{center}
		\includegraphics[width=\linewidth]{finalBlockDiagram.png}
	\end{center}
	\caption{Final Labview block diagram}
	\label{fig:finalDiagram}
\end{figure}

A sample of the front panel during operation is shown in \figref{frontPanel}.
There are two things to notice about the sample plot. First, the smoothing of
the noised inputs for the system. While both tempurature and pressure sensor
inputs show a busy signal with lots of noise the filters from the noise-filter
diagrams produce a final smooth curve that is useful for analysis. Second, the
response behaves as expected. The curve in the middle is indeed behaving as
mathematically expected: below pressure and above the tempurature plots.

\begin{figure}[h]
	\begin{center}
		\includegraphics[width=\linewidth]{frontPanel}
	\end{center}
	\caption{Sample Front Panel During Operation}
	\label{fig:frontPanel}
\end{figure}

% section final_project_design (end)

\end{document}
