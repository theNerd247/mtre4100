\documentclass[main.tex]{subfile}

\begin{document}

\section{Introduction} 
\label{sec:introduction}

A PID controller is the simplest controller design and is most popular in
industrial settings. PID controllers are only used for single input; single
output systems that are inherently stable - that is for a given bounded input the
system has a bounded output. A PID controller is created by first creating a
system model, attaching a general PID controller, and finally tuning the
controller's to achieve some desired output. 

Because time was limited, the reader should note that the models for this
project are not realistic and were created for the sake of showing the use of
\Labview in control modeling. Our control model will predict the flow rate of a
fluid through a pipe by controlling the tempurature and pressure independently.
Our flow rate is modelled as follows: 

\begin{align}
	Q &= 0.6T+0.4P \label{eq:flowRate}
\end{align}
where $Q$ is the flow rate, $T$ is the tempurature, and $P$ is the pressure of
the fluid. Note that the tempurature readings will have more weight on the
calculated flow rate than the pressure. Therefore we will expect the flow rate
plot to be placed below the pressure plot but above the temperature plot.

To make the controllers simple to design we use third order systems for
tempurature and pressure:

\begin{align}
	T &= a_2D^2\{x_1\} + a_1D\{x_1\} + a_0x_1
	\\P &= b_2D^2\{x_2\} + b_1D\{x_2\} + b_0x_2
\end{align}
and in the corresponding transfer functions:

\begin{align}
	\frac{T(s)}{X_1(s)} &= \frac{1}{a_2s^2+a_1s+a_0}
	\\\frac{P(s)}{X_2(s)} &= \frac{1}{b_2s^2+b_1s+b_0}
\end{align}

Realistically the measured tempurature will have noise introduced into it and
will need filtering. To simulate noise we will use the gausian distribution
noise generation and a Butterworth lowpass filter to obtain the actual
tempurature and pressure signals.

% section introduction (end)

% document text goes here

\end{document}
