\documentclass[a4paper,titlepage]{article}
\usepackage{amsmath}
\begin{document}
	\section{Mathematical Analysis}
	\subsection{Background}
	This lab demonstrates the use of Fourier Analysis and the Fourier Series. The Fourier Series is a tool to represent an arbitrary periodic function as a (possibly infinite) sum of sine and cosine waves.
	
	$$f(t) = a_0 + \sum_{1}^{\infty} a_n cos(n \omega_0 t) + b_n sin(n \omega_0 t)$$ 
	Where $n$ is the index of the harmonic, $\omega_0$ is the fundamental frequency of the function (i.e. $\frac{1}{T}$), and the constants $a_0$, $a_n$, and $b_n$ are found as follows.
	
	$$a_0 = \frac{1}{T} \int_{0}^{T}f(t)dt$$
	$$a_n = \frac{2}{T} \int_{0}^{T}f(t) cos(n \omega_0 t)dt$$
	$$b_n = \frac{2}{T} \int_{0}^{T}f(t) sin(n \omega_0 t)dt$$
	\subsection{Application to Square Wave}
	The square wave is defined as
	\[ f(t) =  \begin{cases}
	1 & mT \leq t < (m + \frac{1}{2})T \\
	-1 & (m + \frac{1}{2})T \leq t < (m + 1)T \\
	\end{cases}
	\]
	This necessitates breaking the integrals into pieces to account for the discontinuity in the function. The equations thus become
	
	$$a_0 = \frac{1}{T} \int_{0}^{\frac{1}{2}T}1dt + \frac{1}{T} \int_{\frac{1}{2}T}^{T} -1dt$$ 
	
	$$a_n = \frac{2}{T} \int_{0}^{\frac{1}{2}T}1cos(n \omega_0 t)dt + \frac{2}{T} \int_{\frac{1}{2}T}^{T}-1cos(n \omega_0 t)dt $$
	
	$$b_n = \frac{2}{T} \int_{0}^{\frac{1}{2}T}1sin(n \omega_0 t)dt + \frac{2}{T} \int_{\frac{1}{2}T}^{T} -1sin(n \omega_0 t)dt $$
	
	It is clear by inspection that $a_0$ evaluates to 0. Similarly, the square wave is an odd function, and it can be shown that $a_n = 0$ for all $n$.
	
	
\end{document}