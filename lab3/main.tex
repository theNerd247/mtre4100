\documentclass[a4paper,titlepage]{article}

% DO NOT REMOVE!
\usepackage{subfiles}

%include the packages and styles to use
\usepackage{lab3}
\begin{document}

\labTitle{Lab 3 Digital Aquisition}

% Date is automagically included (see Latex documentation)
\maketitle

\begin{abstract}
	A key concept in process control is how to gather data and understanding how the techniques used to gather that data affect the data's quality and how the data can be used. This experiment illustrates this by using a function generator and an analog to digital converter to show how sampled data differs from the real-world, continuous quantities that they represent. This is important because modern control systems use computers to perform calculations based on the input signal data. Computers operate on data with discrete values which necessitates converting continuously varying input signals (such as electrical voltage) to discrete values for processing. Bearing this in mind, this lab aims to show how sampling rate can affect the quality of data provided to a digital system by sampling known, continuous sources at different rates.
	This lab aims to investigate data acquisition via sinusoidal signals and the
	errors involved therein. We first investigate the affects digital sampling
	frequencies have on the measured data and the use of the Nyquist Frequency.
	Because generated signal have errors within themselves we also investigate the
	effects of amplitude, y-axis offset, and quantization errors in obtaining the
	signal.
\end{abstract}

\tableofcontents
\listoftables
\listoffigures

\pagebreak

% other sections of the document goes here
\subfile{intro.tex}
\subfile{part2.tex}
\subfile{part3.tex}
\subfile{conclusion.tex}

\end{document}
