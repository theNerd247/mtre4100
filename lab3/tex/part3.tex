\documentclass[main.tex]{subfile}

\begin{document}

\section{Offset Errors} 
\label{sec:nyquist_frequency}

\subsection{Null-Offset and Amplitude Error}
\label{sec:amplitude_error}

As shown in \eqref{sampledWave} there are more potential errors present than
aliasing. Firstly there's potential to get the wrong $A_{pp}$ - the
peak-to-peak amplitude. To determine this amplitude error we use the maximum
absolute value in the sampled signal We then compare it to our expected
$A_{pp}$. 

For our experiment we use a sampling rate of $4\dem{kHz}$ to avoid any aliasing
error. \tabref{ampError} shows the amplitude of the data gathered and the error
compared to our expected $A_{pp} = 0.05$ of the data gathered. As seen the
amplitude error does not fall within the acceptable range.

\begin{table}[H]
  \begin{center}
    \caption{Collected data for Amplitude Error}
    \label{tab:ampError}
    \begin{tabular}{lll}
      \\ \toprule
      Maximum Abs. Value & \% Error  & Acceptable Amplitude Range
      \\ \midrule
      $0.054241$ & $8.48\%$ & $0.0495 - 0.0505$
      \\ \bottomrule
    \end{tabular}
  \end{center}
\end{table}

The second type of error we have is null-offset. Simply put this is the error in
the $y$ part of \eqref{sampledWave}. To calculate this we use the average value
of the sampled wave which - theoretically - should be $0$. The average value is
computed using $\frac{1}{t_2-t_1}\int_{t_1}^{t_2}{f(t) dt}$ where the integral
is evaluated using the trapezoidal method. We then find that the offset error of
the sine wave is $6.334e{-3}$.

To determine the exact sources of error more analysis would need to be
performed. However, with some common sense, we can make a rather educated guess.
First the amplitude error could come from the fact that the sampling frequency
is still to low to accurately depict the sampled wave. This is due to the fact
that the period of the sampling frequency is not at a low enough resolution to
capture the peaks of the sine wave. Thus a maximum amplitude seen by the sampled
data would be lower than that of the actual sine wave produced. 

Offset errors could be generated from improper grounding of the sinewave or the
sampling circuit. Also, limited sampling frequencies could produce an offset
error - as the offset is calculated using the amplitude of the sine wave. To
reduce these errors post data analysis could be performed to interpolate the
actual sine wave from the sampled data. Of course this would require a high
enough resolution such that the interpolation would accurately represent the
measured signal.

% subsection null_offset_error (end)

\end{document}
